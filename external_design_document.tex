\documentclass[a4j,titlepage]{jarticle}
\usepackage[dvipdfmx]{graphicx}
\usepackage{ascmac}
\usepackage{url}

\title{居酒屋検索システム\\
外部設計書\\
第0版}
\author{株式会社Spirytus}
\date{\today}

\begin{document}
\maketitle
\tableofcontents

\section{はじめに}

\section{データ定義}
本章では、データベースで管理するデータについて定義します。データベースのテーブルは、表\ref{tables}に示すテーブル群から構成されます。また、これらのテーブル間のつながりを示したER図を別添ファイル「ER図.pdf」に、各テーブルの定義を別添ファイル「データ定義書.pdf」に示します。
​
\begin{table}[!htbp]
\caption{データテーブル一覧}
\label{tables}
\begin{center}
\begin{tabular}{|l|l|l|}\hline
テーブル名 & 内容 & 詳細\\\hline\hline
liquors & お酒 & お酒に関する情報\\\hline
stores & 店舗 & 店舗に関する情報\\\hline
dishes & 料理 & 料理に関する情報\\\hline
ingredients & 食材 & 食材に関する情報\\\hline
resorts & 観光地 & 観光地に関する情報\\\hline
alcoholics & 酒類 & 酒類に関する情報\\\hline
brewers & 酒造 & 酒造に関する情報\\\hline
details & 詳細 & お酒、店舗、郷土料理、観光地の詳細情報\\\hline
liquor\_text\_reviews & お酒の文章レビュー & お酒の文章レビューに関する情報\\\hline
liquor\_number\_reviews & お酒の数値レビュー & お酒の数値レビューに関する情報\\\hline
liquor\_averages & お酒の数値レビューの平均 & お酒の数値レビューの平均に関する情報\\\hline
store\_text\_reviews & 店舗の文章レビュー & 店舗の文章レビューに関する情報\\\hline
store\_number\_reviews & 店舗の数値レビュー & 店舗の数値レビューに関する情報\\\hline
store\_averages & 店舗の数値レビューの平均 & 店舗の数値レビューの平均に関する情報\\\hline
users & ユーザ & ユーザに関する情報\\\hline
store\_users & 店舗ユーザ & 店舗ユーザに関する情報\\\hline
store\_liquors & 店舗にあるお酒 & 店舗とお酒を関連付ける情報\\\hline
store\_dishes & 店舗にある料理 & 店舗と料理を関連付ける情報\\\hline
desh\_ingredients & 料理の食材 & 料理と食材を関連付ける情報\\\hline
liquor\_details & お酒の詳細 & お酒と詳細を関連付ける情報\\\hline
store\_details & 店舗の詳細 & 店舗と詳細を関連付ける情報\\\hline
local\_dishes\_details & 郷土料理の詳細 & 郷土料理と詳細を関連付ける情報\\\hline
resort\_details & 観光地の詳細 & 観光地と詳細を関連付ける情報\\\hline
\end{tabular}
\end{center}
\end{table}

\section{ユーザインタフェース設計}

\section{ネットワーク設計}

\section{セキュリティ対策}
\subsection{不正アクセス対策}
ファイアウォールの導入により、不正アクセスが行われることを防ぎます。
また、侵入検知システム(IDS)を導入することで、
不正アクセスの兆候を検知し、早めの対応を行えるようにします。
\subsection{暗号化通信}
サーバ・クライアント間の通信は全てSSL通信を利用します。
これにより、通信内容の盗聴されることで情報が外部に漏洩することを防ぎます。
また、ユーザからの信頼性を向上させるために、SSL証明書を購入およびインストールします。
\subsection{ユーザ認証}
このシステムでは、店舗情報の登録および更新やユーザによるレビュー投稿機能を、
その権限のないユーザが行うことを防ぐために、ユーザ認証を行います。
店舗関係者とユーザのログイン時は、登録されたe-mailアドレスとパスワードの組み合わせにより認証を行います。
また、パスワードをハッシュ化してからデータベースに保存することで、
データベース内の情報が外部に漏洩してもパスワードが第三者に知られることを防ぎます。

\section{障害対策}
\subsection{ソフトウェアの障害対策}
データベースの障害対策として、メインで使用するHDD以外のHDDに、
データベースのバックアップを保存します。
毎日深夜0時に自動でバックアップを行い、
障害発生時にはこのバックアップデータからリストアし、復元します。
\subsection{許容停止時間}
障害発生によりシステムが機能しない場合、
直ちに再稼働のための手続きを行います。
システムが停止してから再稼働までの許容停止時間は48時間とします。

\section{ソフトウェア品質}
システムに組み込むプログラムは全て、テストを通過したもののみを扱います。
これにより常に一定の品質を確保します。
システムにバグが発見された場合には、直ちに修正しシステムをアップデートし、
必要に応じてテストの内容を更新します。




\end{document}
