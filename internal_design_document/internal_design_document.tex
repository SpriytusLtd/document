\documentclass[a4j,titlepage]{jarticle}
\usepackage[dvipdfmx]{graphicx}
\usepackage{ascmac}
\usepackage{url}
\usepackage{multirow}


\title{呑兵衛土佐巡\\
内部設計書\\
第1版}
\author{株式会社Spirytus}
\date{\today}

\begin{document}
\maketitle
\tableofcontents

\clearpage

\section{概要}
\subsection{動作環境}
動作環境は、以下のブラウザを推奨しています。
\begin{itemize}
\item Chrome 47.0
\item Safari 5.1.10
\item Opera 33.0
\item IE 11.0
\item Fireforx 42.0
\end{itemize}

\subsection{開発環境}
開発環境は、以下のOSを使用しています。
\begin{itemize}
\item centOS 6.5
\item Mac OSX
\end{itemize}

\subsection{使用言語}
本システムでは、以下の言語を使用し開発しています。
\begin{itemize}
\item HTML5
\item CSS3
\item JavaScript
\item Ruby 2.2.3
\end{itemize}

\subsection{使用フレームワーク}
本システムでは、以下のフレームワークを使用しています。
\begin{itemize}
\item Bootstrap 3.3.6
\item jQuery 2.1.4
\item Ruby on Rails 4.2.4
\end{itemize}

\subsection{モジュール一覧}
本システムを構成するモジュールは、以下の通りです。
\begin{enumerate}
\item [SEARCH] : データベースから検索条件にあった情報を取得するモジュール
\item [REGISTER] : データベースに情報を登録するモジュール
\item [LOGIN] : ログインする処理を行なうモジュール
\item [CHECK] : データベースから情報を取得する際、データが正しいか確認するモジュール
\item [DIRECT MAIL] : メールを送信するモジュール
\item [DELETE] : データベースに登録している情報を削除するモジュール
\item [ALGORITHM] : 計算するモジュール
\item [MULTI MODULE] : 複数のモジュールを組み合わせるモジュール
\item [ERROR] : エラーが起こったときに行なうモジュール
\end{enumerate}

\clearpage

\subsection{モジュール定義}
各言語において実現したい機能、および共通の機能は以下の通りです。
\subsubsection{CSS}
\begin{description}
\item[ロールオーバー]~\\
別画面へ遷移するボタンや画像にマウスポインタを当てたときに、色や画像が切り替わる機能を適用します。
この機能を用いることで、別画面に遷移するボタンや画像を、ユーザが簡単に判断することができます。
CSSより、background-imageの切り替えを行うことで、ロールオーバーを実現します。
\end{description}

\subsubsection{JavaScript}
\begin{description}
\item [プルダウン]~\\
ホーム画面の各タブ内において、メニューの階層をまとめて表示することが可能となります。
この機能を用いることで、ユーザが目的のページまで容易に遷移することができます。
JQueryよりhover関数を呼び出し、slideDown()とslideUp()を呼び出すことでプルダウンを実現します。

\item [画像の自動切り替え(スライド)]~\\
画面内で画像を表示させる際、画像の切り替えにスライド効果を適用します。
この機能を用いることで、いくつかの情報(画像)を自動で出力することができるため、
ユーザによりわかりやすく情報を提供することができます。
JQueryよりslider関数を呼び出すことで、画像の自動切り替え(スライド)を実現します。

\item[画像の自動切り替え(フェード)]~\\
上記と同様で、画像の切り替えにフェード効果を適用します。
JQueryよりsetInterval関数を呼び出して、
fadeOutとfadeInを呼び出すことで画像の自動切り替え(フェード)を実現します。

\item[星での評価]~\\
ユーザが店舗へのレビューをする際、店舗の評価を星の数によって決めることができます。
この機能を用いることで、ユーザが簡単に店舗の評価を決めることができます。
JQueryよりraty関数を呼び出すことで、星での評価を実現します。

\item[タブ切り替え]~\\
検索画面において、別画面に遷移することなく検索項目を切り替えることができます。
この機能を用いることで、検索項目を切り替える際にページを読み込む時間がなくなり、ユーザが検索したい項目をより早く選択することができます。
JQueryより、removeClass関数とaddClass関数を呼び出すことで、タブ切り替えを実現します。

\end{description}

\newpage
\section{ルーティング}
本システムの各ページへのルーティングと対応するコントローラは、表\ref{routing}の通りです。

\begin{table}[!htbp]
\caption{ルーティング一覧}
\label{routing}
\small
\begin{center}
\begin{tabular}{|l|l|l|p{4cm}|}\hline
ページ内容 & ルーティング & HTTPメソッド & 対応するコントローラ\\\hline\hline
トップページ & / & GET & \multirow{5}{*}{Indexesコントローラ} \\\cline{1-3}
検索(お酒) & /drink?検索条件 & GET & \\\cline{1-3}
検索(食材) & /food?検索条件 & GET & \\\cline{1-3}
検索(店舗) & /store?検索条件 & GET &  \\\cline{1-3}
検索(語句) & /search?検索語句 & GET & \\\hline
\multirow{2}{*}{お酒(詳細)} & \multirow{2}{*}{/:drink\_name}
& GET & \multirow{4}{*}{Drinksコントローラ} \\\cline{3-3}
 & & POST &  \\\cline{1-3}
\multirow{2}{*}{お酒(レビュー)} & \multirow{2}{*}{/:drink\_name/review}
& GET & \\\cline{3-3}
 & & POST & \\\hline
\multirow{2}{*}{店舗(詳細)} & \multirow{2}{*}{/:store\_name}
& GET & \multirow{4}{*}{Storesコントローラ} \\\cline{3-3}
 & & POST &  \\\cline{1-3}
\multirow{2}{*}{店舗(レビュー)}& \multirow{2}{*}{/:store\_name/review}
& GET &  \\\cline{3-3}
 & & POST &\\\hline
検索(観光地等) & /resort?検索条件 & GET & Resortsコントローラ \\\hline
\multirow{2}{*}{ユーザ登録} & \multirow{2}{*}{/registration\_user}
& GET & \multirow{12}{*}{Userコントローラ} \\\cline{3-3}
 & & POST & \\\cline{1-3}
 \multirow{2}{*}{ログイン(ユーザ)} & \multirow{2}{*}{/login\_user}
 & GET & \\\cline{3-3}
 & & POST & \\\cline{1-3}
マイページ(トップ) & /:user\_name
& GET & \\\cline{1-3}
\multirow{2}{*}{マイページ(お気に入り)} & /:user\_name/favorites
& GET & \\\cline{2-3}
 & /:user\_name/favorites/:favorite\_id & DELETE & \\\cline{1-3}
\multirow{2}{*}{マイページ(レビュー)} & /:user\_name/reviews
& GET & \\\cline{2-3}
 & /user\_name/reviews/:review\_id & DELETE & \\\cline{1-3}
\multirow{3}{*}{マイページ(設定)} & \multirow{3}{*}{/:user\_name/config}
& GET & \\\cline{3-3}
 & & POST & \\\cline{3-3}
 & & DELETE & \\\hline
 \multirow{2}{*}{店舗登録申請} & \multirow{2}{*}{/store/request}
 & GET & \multirow{13}{*}{Storeコントローラ} \\\cline{3-3}
  & & POST &\\\cline{1-3}
 \multirow{2}{*}{店舗登録} & \multirow{2}{*}{/store/registration}
 & GET &\\\cline{3-3}
  & & POST &\\\cline{1-3}
 \multirow{2}{*}{ログイン(店舗)} & \multirow{2}{*}{/login\_store}
 & GET & \\\cline{3-3}
  & & POST & \\\cline{1-3}
\multirow{3}{*}{店舗マイページ(編集)}& \multirow{3}{*}{/store/:store\_name}
& GET & \\\cline{3-3}
 & & POST & \\\cline{3-3}
 & & DELETE & \\\cline{1-3}
店舗マイページ(レビュー) & /store/:store\_name/review
& GET & \\\cline{1-3}
\multirow{3}{*}{店舗マイページ(設定)} & \multirow{3}{*}{/store/:store\_name/config}
& GET & \\\cline{3-3}
 & & POST & \\\cline{3-3}
 & & DELETE & \\\hline
\end{tabular}
\end{center}
\end{table}

\section{コントローラ}
本システムで作成し、使用するコントローラは表\ref{controller}の通りです。

\begin{table}[!htbp]
\caption{コントローラ一覧}
\label{controller}
\small
\begin{center}
\begin{tabular}{|l|p{5cm}|}\hline
コントローラ名 & 詳細 \\\hline\hline
\multirow{2}{*}{Indexes} & トップページ、検索に関するコントローラです。 \\\hline
Users & ユーザに関するコントローラです。 \\\hline
Stores & 店舗に関するコントローラです。 \\\hline
Drinks & お酒に関するコントローラです。 \\\hline
Resorts & 観光地に関するコントローラです。 \\\hline
\end{tabular}
\end{center}
\end{table}

\end{document}
