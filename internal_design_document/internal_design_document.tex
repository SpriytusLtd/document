\documentclass[a4j,titlepage]{jarticle}
\usepackage[dvipdfmx]{graphicx}
\usepackage{ascmac}
\usepackage{url}


\title{呑兵衛土佐巡\\
内部設計書\\
第0版}
\author{株式会社Spirytus}
\date{\today}

\begin{document}
\maketitle
\tableofcontents

\clearpage

\section{概要}
\subsection{動作環境}
動作環境は、以下のブラウザを推奨しています。
\begin{itemize}
\item Chrome
\item Safari
\item Opera
\item IE
\item Fireforx
\end{itemize}

\subsection{開発環境}
開発環境は、以下のOSを使用しています。
\begin{itemize}
\item Unix 6.5
\item Mac X
\end{itemize}

\subsection{使用言語}
本システムでは、以下の言語を使用し開発しています。
\begin{itemize}
\item HTML
\item CSS
\item JavaScript
\item Ruby
\end{itemize}

\subsection{使用フレームワーク}
本システムでは、以下のフレームワークを使用しています。
\begin{itemize}
\item Bootstrap
\item jQuery
\item Ruby on Rails
\end{itemize}

\subsection{モジュール一覧}
本システムを構成するモジュールは、以下の通りです。
\begin{enumerate}
\item [SE] : データベースから検索条件にあった情報を取得するモジュール
\item [RE] : データベースに情報を登録するモジュール
\item [IN] : ログインする処理を行なうモジュール
\item [CH] : データベースから情報を取得する際、データが正しいか確認するモジュール
\item [DM] : メールを送信するモジュール
\item [DE] : データベースに登録している情報を削除するモジュール
\item [AL] : 計算するモジュール
\item [MM] : 複数のモジュールを組み合わせるモジュール
\item [ER] : エラーが起こったときに行なうモジュール
\end{enumerate}
\end{document}
