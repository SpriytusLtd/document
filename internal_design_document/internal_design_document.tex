\documentclass[a4j,titlepage]{jarticle}
\usepackage[dvipdfmx]{graphicx}
\usepackage{ascmac}
\usepackage{url}
\usepackage{multirow}


\title{呑兵衛土佐巡\\
内部設計書\\
第0版}
\author{株式会社Spirytus}
\date{\today}

\begin{document}
\maketitle
\tableofcontents

\clearpage

\section{概要}
\subsection{動作環境}
動作環境は、以下のブラウザを推奨しています。
\begin{itemize}
\item Chrome 47.0
\item Safari 5.1.10
\item Opera 33.0
\item IE 11.0
\item Fireforx 42.0
\end{itemize}

\subsection{開発環境}
開発環境は、以下のOSを使用しています。
\begin{itemize}
\item centOS 6.5
\item Mac OSX
\end{itemize}

\subsection{使用言語}
本システムでは、以下の言語を使用し開発しています。
\begin{itemize}
\item HTML
\item CSS
\item JavaScript
\item Ruby
\end{itemize}

\subsection{使用フレームワーク}
本システムでは、以下のフレームワークを使用しています。
\begin{itemize}
\item Bootstrap
\item jQuery
\item Ruby on Rails
\end{itemize}

\subsection{モジュール一覧}
本システムを構成するモジュールは、以下の通りです。
\begin{enumerate}
\item [SEARCH] : データベースから検索条件にあった情報を取得するモジュール
\item [REGISTER] : データベースに情報を登録するモジュール
\item [LOGIN] : ログインする処理を行なうモジュール
\item [CHECK] : データベースから情報を取得する際、データが正しいか確認するモジュール
\item [DIRECT MAIL] : メールを送信するモジュール
\item [DELETE] : データベースに登録している情報を削除するモジュール
\item [ALGORITHM] : 計算するモジュール
\item [MULTI MODULE] : 複数のモジュールを組み合わせるモジュール
\item [ERROR] : エラーが起こったときに行なうモジュール
\end{enumerate}

\clearpage

\subsection{モジュール定義}
各言語において実現したい機能、および共通の機能は以下の通りです。
\subsubsection{CSS}


\subsubsection{JavaScript}
\begin{description}
\item [プルダウン]~\\
ホーム画面の各タブ内において、メニューの階層をまとめて表示することが可能となります。
この機能を用いることで、ユーザが目的のページまで容易に遷移することができます。
JQueryよりhover関数を呼び出し、slideDown()とslideUp()を呼び出すことでプルダウンを実現します。

\item [画像の自動切り替え(スライド)]~\\
画面内で画像を表示させる際、画像の切り替えにスライド効果を適用します。
この機能を用いることで、いくつかの情報(画像)を自動で出力することができるため、
ユーザによりわかりやすく情報を提供することができます。
JQueryよりslider関数を呼び出すことで、画像の自動切り替え(スライド)を実現します。

\item[画像の自動切り替え(フェード)]~\\
上記と同様で、画像の切り替えにフェード効果を適用します。
JQueryよりsetInterval関数を呼び出して、
fadeOutとfadeInを呼び出すことで画像の自動切り替え(フェード)を実現します。
\end{description}

\newpage
\section{ルーティング}
本システムの各ページへのルーティングは、表\ref{routing1}及び表\ref{routing2}の通りです。

\begin{table}[!htbp]
\caption{ルーティング一覧(1/2)}
\label{routing1}
\small
\begin{center}
\begin{tabular}{|l|l|l|p{4cm}|}\hline
ページ内容 & ルーティング & HTTPメソッド & 詳細\\\hline\hline
トップページ & / & GET & トップページを表示します。 \\\hline
\multirow{2}{*}{検索(お酒)} & \multirow{2}{*}{/search/drink?検索条件}
& \multirow{2}{*}{GET} & 指定された検索条件に合致するお酒を表示します。\\\hline
\multirow{2}{*}{検索(食材)} & \multirow{2}{*}{/search/food?検索条件}
& \multirow{2}{*}{GET} & 指定された検索条件に合致する店舗を表示します。\\\hline
\multirow{2}{*}{検索(観光地等)} & \multirow{2}{*}{/search/resort?検索条件}
& \multirow{2}{*}{GET} & 指定された検索条件に合致する店舗を表示します。\\\hline
\multirow{2}{*}{検索(店舗)} & \multirow{2}{*}{/search/store?検索条件}
& \multirow{2}{*}{GET} & 指定された検索条件に合致する店舗を表示します。 \\\hline
\multirow{2}{*}{検索(語句)} & \multirow{2}{*}{/search?検索語句}
& \multirow{2}{*}{GET} & 指定された検索語句に合致する店舗を表示します。 \\\hline
\multirow{4}{*}{お酒(詳細)} & \multirow{4}{*}{/detuil/:drink\_name}
& \multirow{2}{*}{GET} & お酒の詳細ページを表示します。 \\\cline{3-4}
 & & \multirow{2}{*}{POST} & お気に入り登録情報を送信します。 \\\hline
\multirow{3}{*}{お酒(レビュー)} & \multirow{3}{*}{/detuil/:drink\_name/review}
& \multirow{2}{*}{GET} & お酒に投稿されたレビューを表示します。\\\cline{3-4}
 & & POST & レビューを送信します。 \\\hline
\multirow{4}{*}{店舗(詳細)} & \multirow{4}{*}{/store:store\_name}
& \multirow{2}{*}{GET} & 店舗の詳細ページを表示します。\\\cline{3-4}
 & & \multirow{2}{*}{POST} & お気に入り登録情報を送信します。 \\\hline
\multirow{3}{*}{店舗(レビュー)}& \multirow{3}{*}{/store:store\_name/review}
& \multirow{2}{*}{GET} & 店舗に投稿されたレビューを表示します。 \\\cline{3-4}
 & & POST & レビューを送信します。 \\\hline
\multirow{4}{*}{ログイン(ユーザ)} & \multirow{4}{*}{/login\_user}
& \multirow{2}{*}{GET} & ユーザ用のログイン画面を表示します。 \\\cline{3-4}
& & \multirow{2}{*}{POST} & メールアドレスとパスワードを送信します。 \\\hline
\multirow{4}{*}{ログイン(店舗)} & \multirow{4}{*}{/login\_store}
& \multirow{2}{*}{GET} & 店舗用のログインページを表示します。 \\\cline{3-4}
 & & \multirow{2}{*}{POST} & メールアドレスとパスワードを送信します。 \\\hline
\end{tabular}
\end{center}
\end{table}

\begin{table}[!htbp]
\caption{ルーティング一覧(2/2)}
\label{routing2}
\small
\begin{center}
\begin{tabular}{|l|l|l|p{4cm}|}\hline
ページ内容 & ルーティング & HTTPメソッド & 詳細\\\hline\hline
\multirow{2}{*}{マイページ(トップ)} & \multirow{2}{*}{/:user\_name}
& \multirow{2}{*}{GET} & ユーザ用のマイページのトップページを表示します。 \\\hline
\multirow{4}{*}{マイページ(お気に入り)} & \multirow{4}{*}{/:user\_name/favorite}
& \multirow{2}{*}{GET} & ユーザの登録したお気に入り一覧を表示します。 \\\cline{3-4}
 & & \multirow{2}{*}{DELETE} & 削除対象をパラメータで指定します。 \\\hline
\multirow{4}{*}{マイページ(レビュー)} & \multirow{4}{*}{/:user\_name/review}
& \multirow{2}{*}{GET} & ユーザの投稿したレビュー一覧を表示します。 \\\cline{3-4}
 & & \multirow{2}{*}{DELETE} & 削除対象をパラメータで指定します。 \\\hline
\multirow{5}{*}{マイページ(設定)} & \multirow{5}{*}{/:user\_name/config}
& \multirow{2}{*}{GET} & ユーザアカウントの詳細設定ページを表示します。 \\\cline{3-4}
 & & POST & 設定情報を送信します。 \\\cline{3-4}
 & & \multirow{2}{*}{DELETE} & 削除対象をパラメータで指定します。 \\\hline
\multirow{5}{*}{店舗マイページ(編集)} & \multirow{5}{*}{/edit/:store\_name}
& \multirow{2}{*}{GET} & 店舗の 登録情報編集画面を表示します。\\\cline{3-4}
 & & POST & 登録情報を送信します。 \\\cline{3-4}
 & & \multirow{2}{*}{DELETE} & 削除対象をパラメータで指定します。\\\hline
\multirow{2}{*}{店舗マイページ(レビュー)} & \multirow{2}{*}{/edit/:store\_name/review}
& \multirow{2}{*}{GET} & 店舗に投稿されたレビューを表示します。 \\\hline
\multirow{5}{*}{店舗マイページ(設定)} & \multirow{5}{*}{/edit/:store\_name/config}
& \multirow{2}{*}{GET} & 店舗の詳細設定ページを表示します。\\\cline{3-4}
 & & POST & 設定情報を送信します。 \\\cline{3-4}
 & & \multirow{2}{*}{DELETE} & 削除対象をパラメータで表示します。 \\\hline
\multirow{3}{*}{ユーザ登録} & \multirow{3}{*}{/registration\_user}
& \multirow{2}{*}{GET} & ユーザ用の登録ページを表示します。 \\\cline{3-4}
 & & POST & 登録情報を送信します。 \\\hline
\multirow{3}{*}{店舗登録申請} & \multirow{3}{*}{/request}
& \multirow{2}{*}{GET} & 店舗用の登録申請ページを表示します。 \\\cline{3-4}
 & & POST & 登録申請情報を送信します。\\\hline
\multirow{3}{*}{店舗登録} & \multirow{3}{*}{/registration\_store}
& \multirow{2}{*}{GET} & 店舗用の登録ページを表示します。\\\cline{3-4}
 & & POST & 登録情報を送信します。\\\hline
\end{tabular}
\end{center}
\end{table}

\clearpage

\section{モデル}

本システムで作成し、使用するモデルは表\ref{model}の通りです。

\begin{table}[!htbp]
\caption{モデル一覧}
\label{model}
\small
\begin{center}
\begin{tabular}{|l|l|l|p{5cm}|}\hline
モデル名 & \multicolumn{2}{|l|}{リレーション} & 詳細 \\\hline\hline
\multirow{2}{*}{User} & Review & 1:N & \multirow{2}{*}{ユーザに関するモデルです。}\\\cline{2-3}
 & Favorite & 1:N & \\\hline
\multirow{4}{*}{Store} & Drink & N:M & \multirow{4}{*}{店舗に関するモデルです。} \\\cline{2-3}
 & Dish & N:M & \\\cline{2-3}
 & Review & 1:N & \\\cline{2-3}
 & Favorite & 1:N & \\\hline
\multirow{5}{*}{Drink} & Store & N:M & \multirow{5}{*}{お酒に関するモデルです。} \\\cline{2-3}
 & Brewer & N:1 & \\\cline{2-3}
 & Alcoholic & N:1 & \\\cline{2-3}
 & Review & 1:N & \\\cline{2-3}
 & Favorite & 1:N & \\\hline
\multirow{2}{*}{Dish} & Store & N:M & \multirow{2}{*}{料理に関するモデルです。} \\\cline{2-3}
 & Ingredient & N:M & \\\hline
Ingredient & Dish & N:M & 食材に関するモデルです。 \\\hline
Brewer & Drink & 1:N & 酒蔵に関するモデルです。 \\\hline
Alcoholic & Drink & 1:N & 酒類に関するモデルです。 \\\hline
\multirow{3}{*}{Review} & User & N:1 & \multirow{3}{*}{レビューに関するモデルです。} \\\cline{2-3}
 & Store & N:1 & \\\cline{2-3}
 & Drink & N:1 & \\\hline
\multirow{3}{*}{Favorite} & User & N:1 & \multirow{3}{*}{お気に入りに関するモデルです。} \\\cline{2-3}
 & Store & N:1 & \\\cline{2-3}
 & Drink & N:1 & \\\hline

Resort & & & 観光地に関するモデルです。 \\\hline

\end{tabular}
\end{center}
\end{table}

\end{document}
