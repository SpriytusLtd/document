\documentclass[a4j,titlepage]{jarticle}
\usepackage[dvipdfmx]{graphicx}
\usepackage{ascmac}
\usepackage{url}
\subsection{モジュール定義}
 各言語において実現したい機能、および共通の機能は以下の通りです。

\subsubsection{HTML}
\begin{description}

\item [meta]~\\
metaタグは以下の通りです。

\begin{itemize}
<meta charset = "utf-8">
\end{itemize}


\item [link]~\\
linkタグは以下の通りです。

\begin{itemize}
<link rel="stylesheet" href=".css" type="text/css">
\end{itemize}


\item [script]~\\
scriptタグは以下の通りです。

\begin{itemize}
<script type="text/javascript" src=".js"></script>
\end{itemize}
\end{description}

\subsubsection{CSS}

\item [フォント]~\\
フォントは以下の2種類です。

\begin{itemize}
\item serif
\item cursive
\end{itemize}

\item [フォントサイズ]~\\
フォントは%指定で以下の3種類です。

\begin{itemize}
\item お酒、食材、観光地、店舗の説明文を100%
\item 題名を150%
\item 題名の赤文字200%
\end{itemize}


\item [class]~\\
汎用的に使用するclassは以下の3種類です。

\begin{itemize}
\item ロゴ(クリックするとTopページに戻る)
\item デザイン
\item ページTopへ戻るボタン
\end{itemize}


\end{document}
