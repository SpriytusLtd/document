\documentclass[a4j,titlepage]{jarticle}
\usepackage[dvipdfmx]{graphicx}
\usepackage{ascmac}
\usepackage{url}
\begin{document}
\subsection{モジュール定義}
 各言語において実現したい機能、および共通の機能は以下の通りです。

\subsubsection{HTML}
\begin{description}
\item [title]~\\
画面のタイトルは以下の通りです。

\begin{itemize}
\item 呑兵衛土佐巡
\item 「酒種 味 アルコール度数」のお酒 呑兵衛土佐巡
\item 高知の「食材の種類名」 呑兵衛土佐巡
\item 「酒名or食材名or郷土料理名or観光地名」に該当する店舗 呑兵衛土佐巡
\item 「検索ワード」の検索結果 呑兵衛土佐巡
\item 「酒名」詳細情報 呑兵衛土佐巡
\item 「酒名」レビュー 呑兵衛土佐巡
\item 「店舗名」詳細情報 呑兵衛土佐巡
\item 「店舗名」レビュー 呑兵衛土佐巡
\item 高知の観光地一覧 呑兵衛土佐巡
\item ユーザ登録 呑兵衛土佐巡
\item ユーザログイン 呑兵衛土佐巡
\item マイページ 呑兵衛土佐巡
\item マイページ お気に入り一覧 呑兵衛土佐巡
\item マイページ お気に入り削除 呑兵衛土佐巡
\item マイページ レビュー一覧 呑兵衛土佐巡
\item マイページ レビュー削除 呑兵衛土佐巡
\item マイページ設定 呑兵衛土佐巡
\item 店舗登録申請 呑兵衛土佐巡
\item 店舗登録 呑兵衛土佐巡
\item 店舗ログイン 呑兵衛土佐巡
\item 店舗マイページ 呑兵衛土佐巡
\item 店舗マイページ レビュー一覧 呑兵衛土佐巡
\item 店舗マイページ設定 呑兵衛土佐巡
\end{itemize}

\item [meta]~\\
metaタグは以下の通りです。

\begin{itemize}
\item $<meta\ charset = "utf-8">$
\end{itemize}


\item [link]~\\
linkタグは以下の通りです。

\begin{itemize}
\item $<link\ rel="stylesheet"\ href=".css"\ type="text/css">$
\end{itemize}


\item [script]~\\
scriptタグは以下の通りです。

\begin{itemize}
\item $<script\ type="text/javascript"\ src=".js"></script>$
\end{itemize}
\end{description}

\subsubsection{CSS}
\begin{description}
\item [フォント]~\\
フォントは以下の2種類です。

\begin{itemize}
\item serif
\item cursive
\end{itemize}

\item [フォントサイズ]~\\
フォントは\%指定で以下の3種類です。

\begin{itemize}
\item お酒、食材、観光地、店舗の説明文を100\%
\item 題名を150\%
\item 題名の赤文字200\%
\end{itemize}


\item [class]~\\
汎用的に使用するclassは以下の3種類です。

\begin{itemize}
\item ロゴ(クリックするとTopページに戻る)
\item デザイン
\item ページTopへ戻るボタン
\end{itemize}
\end{description}
\end{document}
